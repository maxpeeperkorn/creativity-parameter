\documentclass[a4paper]{article}

\begin{document}

\section{Appendix}

\subsection{A Questionnaire}\label{app:questionnaire}
The questionnaire consists of three parts. The first part contains the questions for the creativity evaluation, followed by questions regarding the participants' creative interests and their stance towards generative AI.
Questions are presented on a five-point scale unless stated otherwise.
Note that in the survey, we use `prototype story' to denote the exemplar story. However, the principle of the evaluation task remains identical.

\subsubsection{Evaluation Questions}
\begin{itemize}
    \item To what extent is the above story novel compared to the prototype story?
    \item To what extent is the above story typical compared to the prototype story?
    \item To what extent is the above story consistent (e.g. sentences are connected and characters, events and actions are used appropriately)?
    \item To what extent is it easy or difficult to follow and understand the above story?
\end{itemize}

\subsubsection{Post-Evaluation Questions}
\begin{itemize}
    \item[\tiny$\bullet$] [open question] Can you describe your experience with writing (short) stories, if any?
    \item[\tiny$\bullet$] [open question] Can you describe your experience with reviewing things you have read, if any?
    \item How familiar are you with generative AI?
    \item How comfortable are you with generative AI being used in creative works, such as art, music,
and writing?
    \item[\tiny$\bullet$] [open question] Do you have any comments on how you approached evaluating the stories in this study?
\end{itemize}

\subsubsection{Definitions}
\begin{itemize}
\item \textbf{Novelty}: A story is novel if it has something new or unusual. 
\item \textbf{Typicality}: A story is considered typical if it exhibits essential characteristics of a story.
\item \textbf{Cohesion}: A story is cohesive if the use of words and sentences are grammatically consistent. Characters, events, and actions are used appropriately.
\item \textbf{Coherence}: A story is coherent if it is easy to follow and understand its meaning.
\end{itemize}
\vfill

\newpage
\subsection{B The Exemplar Story}\label{app:exemplar-story}

\begin{story}[!h]
\sffamily
\small
\hrule
\vspace{1ex}
Once upon a time, there was a young girl named Maria. She lived in a small village nestled in the rolling hills of the countryside. Maria was a curious and adventurous child, always eager to explore the world around her.
\vspace{1ex}

\hspace{10pt} One day, while wandering through the village, Maria stumbled upon a hidden path she had never seen before. The path was overgrown with weeds and vines, and it looked as though it hadn't been used in years.
\vspace{1ex}

\hspace{10pt} Without hesitation, Maria decided to follow the path. She pushed aside the branches and vines, and began to walk down the winding trail. The path led her through the forest, and the trees grew taller and closer together as she walked.
\vspace{1ex}

\hspace{10pt} As she continued on, Maria began to notice strange markings on the trees. They were unlike anything she had ever seen before - strange symbols and patterns that seemed to glow in the fading light of day.
\vspace{1ex}

\hspace{10pt} Maria's curiosity grew with each step, and she quickened her pace. She knew that she had to find out what these markings meant.
\vspace{1ex}

\hspace{10pt} After a while, the path opened up into a clearing. In the center of the clearing stood an enormous tree, its trunk twisted and gnarled with age. The symbols and patterns that Maria had seen on the other trees were etched into the bark of this tree as well, but they were even more intricate and complex.
\vspace{1ex}

\hspace{10pt} Maria approached the tree, her heart pounding with excitement. As she reached out to touch the bark, a voice boomed from within the tree.
\vspace{1ex}

\hspace{10pt} ``Who dares to disturb the resting place of the ancient ones?'' it said.
\vspace{1ex}

\hspace{10pt} Maria was startled, but she stood her ground. ``I'm just a curious child,'' she said. ``I didn't mean to disturb anyone.''
\vspace{1ex}

\hspace{10pt} There was a long silence, and then the voice spoke again. ``You have been chosen to receive a great gift,'' it said. ``Are you willing to accept it?''
\vspace{1ex}

\hspace{10pt} Maria nodded eagerly, and the tree began to glow with a soft, golden light. The symbols and patterns on the bark began to move and shift, and Maria felt a strange energy coursing through her body.
\vspace{1ex}

\hspace{10pt} When the light faded, Maria found that she had been given the power to communicate with the spirits of the forest. From that day on, she used her gift to help the villagers and the forest creatures, and she lived a long and happy life filled with wonder and adventure.
\vspace{1ex}

\hspace{10pt} The end.
\vspace{1ex}
\hrule
\vspace{1ex}
\rmfamily
\caption{The exemplar object for \autoref{fig:exact-llama-2-prompt} using Llama 70B-chat.}\label{narrative:the-prototype-story}
\end{story}


\end{document}